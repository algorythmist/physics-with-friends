\documentclass[]{article}

%opening
\title{Road to Reality\\Chapters 7-8}
\author{Dimitri Papaioannou}

\usepackage{amsmath}
\usepackage{mathtools}

\begin{document}

\maketitle


\section*{Chapter 7}

Review of complex analysis

\begin{itemize}
\item If a complex derivative exists then it is continuous. 
\item If the first derivative exists, then all derivatives exist, and a Taylor expansion exists.
\item A function with a Taylor expansion is called \textit{analytic}. The definition of analytic goes beyond complex functions to describe functions in other domains that have Taylor expansions. A complex analytic function is also called \textit{holomorphic}.
\item Write $f(z)= f(x+iy) = u(x,y)+iv(x,y)$. Then u and v are conjugate harmonic and satisfy the Cauchy Riemann equations. (Problem 10.12)
\end{itemize}

\subsection*{7.1}
For nonzero n, $z^n$ has an anti-derivative $z^{n+1}/(n+1)$
Integrated over the unit circle we have $e^{i(n+1)\theta}/(n+1)$
evaluated at $2\pi$ and $0$, which yields 0. 

\subsection*{7.2}
Substituting the McLaurin series, all terms other than 1/z integrate to 0 (by 7.1) leaving 
$$
\frac{1}{2\pi i} \oint \frac{f(0)}{z}dz = f(0)
$$

because 
$$
\oint \frac{dz}{z} = 2\pi i
$$


\subsection*{7.3}
Using the following relationship as the "definition" of the derivative of an analytic function

\begin{equation}
f^{(n)} := \frac{n!}{2\pi i}\oint \frac{f(z)}{z^{n+1}}dz
\end{equation}

show that the Taylor expansion sums up to $f(x)$:
\begin{equation}
\sum_{n=0}^\infty a_n z^n
\end{equation}

where $a_n = f^{(n)}(0)/n!$.

\textbf{Solution}
Evaluate the series at $z = p$
\begin{eqnarray}
\sum_{n=0}^\infty a_n p^n &= \frac{n!}{2\pi i} \sum \oint \frac{f(z)}{z^{n+1}} p^n dz
\end{eqnarray}
I am going to assume without proof that I can interchange integration and summation

\begin{eqnarray}
\sum_{n=0}^\infty a_n p^n &= 
\frac{1}{2\pi i} \oint \frac{f(z)}{z} \sum (\frac{p}{z})^n dz \\
  &= \frac{1}{2\pi i} \oint \frac{f(z)}{z} \frac{z}{z-p} dz \\
  &= \frac{1}{2\pi i} \oint \frac{f(z)}{z-p}dz\\
  &= f(p)
\end{eqnarray}
where the last step follows from the shifted version of the Cauchy formula.

\subsection*{7.4}
This amounts to showing that a contour integral can be broken into a sum of contour integrals surrounding each pole. Then the result follows from the Cauchy formula:

$$
\oint \frac{h(z)}{(z-p)^n} dz = \frac{2\pi i}{(n-1)!} h^{(n-1)}(p)
$$

\subsection*{7.5}
Show that $\int_0^{\infty} x^{-1}sin(x)dx = \pi/2$ \\

\textbf{Solution}:
Following the hint, I will integrate the function  $z^{-1}e^{iz}$ along the suggested path.

The function has a pole at 0 and the value of the residue there is 1.

Therefore we have 

\begin{eqnarray}
\int_{-R}^{-\epsilon}\frac{e^{ix}}{x}dx +\int_{\epsilon-semicircle } \frac{e^{iz}}{z}dz +\\
\int_{\epsilon}^R\frac{e^{ix}}{x}dx +\int_{R-semicircle } \frac{e^{iz}}{z}dz &= 0
\end{eqnarray}
%\int_{-R}^{R}\frac{cos(x)}{x-a}dx +i\int_{-R}^{R}\frac{sin(x)}{x-a}dx +\int_{semicircle} \frac{e^{iz}}{z-a}dz &= 2\pi ie^{-a}


First we need to show that the integral over the R-semicircle goes to 0 as R goes to infinity. This can be shown by direct calculation that shows the anti-derivative of 
$e^{iRe^{i\theta}}$ is a bounded function divided by R (TODO: supply details)
 
The integral over the $\epsilon$ semicircle is half of the integral around the pole (going around clockwise), so it evaluates to $-i*pi$

Therefore we have:
\begin{equation}
\int_{-R}^{-\epsilon}\frac{e^{ix}}{x}dx + \int_{\epsilon}^R\frac{e^{ix}}{x}dx = i\pi
\end{equation}

Now we have $frac{e^{ix}}{x} = \frac{cos(x)}{x} + i\frac{sin(x)}{x}$.

The first function is anti-symmetric, so the two opposite integrals cancel each other.
The other function is symmetric so the two integrals are the same. Putting this together we get:

\begin{equation}
2i\int_{0}^{\infty}\frac{sin(x)}{x}dx = i\pi
\end{equation}



\subsection*{7.6}
Sow that $\sum_{n=0}^\infty \frac{1}{n^2} = \frac{\pi^2}{6}$

\textbf{Solution}: TODO

\subsection*{7.7}
What is the power series for $1/z$ around p

\textbf{Solution}
$$\sum_{n=0}^\infty (-1)^n p ^{-(n+1)} (z-p)^n$$

\subsection*{7.8}
Derive the power series for $lnz$ around 1

\textbf{Solution}: 
$dlogz/dz = 1/z$. Subsequent powers of $1/z$ yield $(-1)^{n-1} \frac{z^{n}}{n}$.
All these evaluated at $z = 1$ yield the coefficients $(-1)^{n-1}\frac{1}{n}$.

\section*{Chapter 8}

\subsection*{8.1}
Show that the Riemann surface $z^a$ joins back after $n$ turns when $a = m/n$ is rational.

\textbf{Solution?}
 $z^a$  has a branch point at 0. If  $a = m/n$ then
 $(z^{a})^{kn} = z^{km}$ which is analytic on the whole complex plane. 
 
\subsection*{8.2}
Figure out the topology of the Riemann surface $(1-z^4)^{1/2}$

\textbf{TODO}

\subsection*{8.3}
The Riemann surface for $logz$ is topologically equivalent to the Riemann sphere with 
a single missing point. The missing point can be unambiguously replaced to yield the entire sphere. Can you see how this comes about?

Hint: Think of the Riemann sphere of the variable $w = logz$.

\textbf{TODO}

\subsection*{8.4}
Show that $z \rightarrow z^{-1}$ maps circles to circles.

\textbf{Solution}
Consider a circle of radius 1 centered at the real number $r$. This circle is given
by $r+e^{i\theta}$.
We can generalize to any radius and any offset by simply multiplying with a complex number $c$ thus causing a rotation and scaling. 

The conformal mapping $z^{-1}$ sends this circle to $\frac{1}{r+e^{i\theta}}$.

After a lot of manipulations we find that this expression is equivalent to the circle given by
$$\frac{r}{r^2-1}+\frac{e^{-i\theta}}{r^2-1}$$

\subsection*{8.5}
Verify that any Mobius transformation can be obtained by a sequence of linear $\rightarrow$ inversion $\rightarrow$ linear transformations.

\textbf{Solution}
Straightforward

\subsection*{8.6}
Check that the two stereographic projections from the north and south poles are related by the transformation $z \rightarrow 1/z$.

\textbf{Solution}
It suffices to show it on a 2D circle. The result can be extended to the whole sphere by a rotation.

Fix a point $\bar{x}$ on the real line.  
The line connecting it to the north pole of the sphere (circle) is given by the equation $y = \frac{-1}{\bar{x}} +1$.
The circle equation is $x^2+y^2 = 1$.
Solving this system we obtain the point on the circle corresponding to $\bar{x}$:

\begin{equation}
x^* = \frac{2\bar{x}}{\bar{x}^2+1}
\end{equation}


\begin{equation}
y^* = \frac{\bar{x}^2-1}{\bar{x}^2+1}
\end{equation}

The antipodal point on the southern hemisphere is given by 
$(x^*, -y^*)$.

We seek the line that connects this point to the northern hemisphere and where it intersects the real line.

We can easily compute the slope of this line to be 
$-\bar{x}$ and the intercept 1. The line intersects the real axis at $0 = -\bar{x}*x + 1$.
Solving for x we get $x = \frac{1}{\bar{x}}$ which is the desired result.


\subsection*{8.7}
Show that the transformation 
$$
t = \frac{z-1}{iz+i}
$$
sends lines parallel to the x-axis to circles in the t-plane.

\textbf{Solution}
TODO: Note that this is the same transform as in  9.5



\subsection*{8.8}
Form a parallelogram on the complex plane by connecting the points (0,1,p,1+p). Then identify opposite edges to form a torus. For different values of p, all such surfaces are topologically equivalent, but may not be \textit{holomorphically} equivalent, in the sense that no holomorhpic mapping can be constructed from one to another.

This question is asking to verify that certain transformations of p, such as 1+p, -p, 1/p, yield holomorphically equivalent surfaces. Furthermore, it asks to find all the special values of p that lead to additional discrete symmetries of the Riemann surface.

\textbf{Discussion of possible Solution}

Must construct a holomorphic mapping (Mobius?) that maps one torus to another. It is not clear to me how to define these mappings along the identified edges. 

For the second part, I suspect choosing p such that the result is a square or a line would lead to additional symmetries, whatever this is supposed to mean.


\end{document}