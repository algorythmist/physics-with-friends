\documentclass[]{article}

%opening
\title{Road to Reality\\ Chapters 4-6}
\author{Dimitri Papaioannou}

\usepackage{amsmath}
\usepackage{mathtools}

\begin{document}

\maketitle

\section*{Chapter 4}

\subsection*{4.1}

[4.1], [4.2], [4.3] is just simple complex algebra.



\subsection*{4.4}
Show that
$$
1+x^2+x^4+x^6+... = (1-x^2)^{-1}
$$

\textbf{Solution}
Assume that the series converges to a function $f(x)$
Then
\begin{eqnarray}
f(x) &= \sum_{n=0}^\infty x^{2n} &\implies \\
f(x) &= 1 + x^2\sum_{n=0}^\infty x^{2n} &\implies \\
f(x) &= 1+x^2f(x) &\implies \\
f(x) &= \frac{1}{1-x^2}
\end{eqnarray}


\subsection*{4.5}
$(1+x^2)^{-1}$ can be obtained from $(1-x^2)^{-1}$ by the transformation $x \rightarrow ix$

\section*{Chapter 5}

\subsection*{5.1, 5.2}
Geometrically, addition of complex numbers is vector addition in $R^2$ 
and multiplication is a combination of scaling and rotation.

In addition, parallelograms degenerate to lines when the two complex numbers are co-linear.
In other words if for $a+ib$ and $c+id$ we have $a/b = c/d$. Exact inverses degenerate to a point.

Similarly, in multiplication triangles degenerate to lines where the numbers are co-linear.

\subsection*{5.3}
Show that multiplication in the complex plane preserves shapes and angles with direct computation and without trigonometry.

TODO

\subsection*{5.4}
This becomes obvious once exponential notation is introduced. Otherwise, it is a boring trigonometry exercise.

\subsection*{5.5}
Show $e^{(a+b)} = e^a e^b$ using the Taylor series expansion.

\textbf{Solution}

\begin{eqnarray}
e^{(a+b)} &= \sum_{n=0}^{\infty}\frac{(a+b)^n}{n!}\\
\label{ea+b}
&= \sum_{n=0}^{\infty}\sum_{k=0}^n \frac{a^k b^{n-k}}{k!(n-k)!}
\end{eqnarray}

On the other hand

\begin{equation}
e^a e^b = \sum_{n=0}^{\infty}\frac{a^n}{n!} \sum_{m=0}^{\infty}\frac{b^m}{m!}
\end{equation}

Introduce a change of variable by setting $m = n-k$. 
Then the new variable $k$ ranges from $n$ (when $m=0$) to $0$ (when $m=\infty$).

\begin{eqnarray}
e^a e^b &= \sum_{n=0}^{\infty}\frac{a^n}{n!} \sum_{k=0}^{n}\frac{b^{n-k}}{(n-k)!}\\
&= \sum_{n=0}^{\infty}\sum_{k=0}^n \frac{a^k b^{n-k}}{k!(n-k)!}
\end{eqnarray}

Which produces the same result as in (\ref{ea+b}).

\subsection*{5.6}
Show that $z+i\pi$ is a logarithm of $-w$.

\textbf{Solution}
\begin{eqnarray}
w = e^z \implies log(-w) = log(-e^z) = log(-1)+log(e^z) \\
= log(e^{i\pi})+z = z + i\pi
\end{eqnarray}


\subsection*{5.8}
Show 
$$
\cos 3\theta = \cos^3 \theta - 3\cos\theta \sin^2 \theta 
$$
$$
\sin 3\theta = 3\sin\theta \cos^2 \theta - \sin^3 \theta
$$
\textbf{Solution}
Expand $e^{3i\theta} = (e^{i\theta})^3$ in real and imaginary parts

\subsection*{5.9}
I do not understand what the plot is plotting. 

\subsection*{5.10}
Resolve the paradox: 
$$
e = e^{1+2\pi i} = (e^{1+2\pi i})^{1+2\pi i} = e^{1+4\pi i-4\pi^2} = e^{1-4\pi^2}
$$

\textbf{Solution}
This is explained by 5.15. Because of the multi-valuedness of the complex power, we cannot willi-nilly conclude that  $w^{ab} = e^{ab \ln w}$.

\subsection*{5.11}
$$
z = lob_b z \implies w = b^z \implies lnw = zlnb \implies z = \frac{lnw}{lnb}
$$
And since we can add $2ki\pi$ to $lnw$, we can add $2ki\pi/lnb$ to z.

\subsection*{5.12}
Why is it allowable to specify $logi = \frac{1}{2}\pi i$?

\textbf{Solution} because $i = e^{i\pi/2}$

\subsection*{5.13}

$$
e^{2\pi n} = e^i \cdot e^{-i2\pi n} = e^i
$$
 
 
\subsection*{5.14}
Multivaluedness of $w^{1/n}$

\textbf{Idea}

Starting from $z^n = w$ and taking the log of both sides we have

$nlogz = logw+2ik\pi$ where the second term results from the multi-valuedness of log.
$logz = (1/n)logw + \frac{2ik\pi}{n} = logw^{1/n} + \frac{2ik\pi}{n}$

From which $z e^{\frac{2ik\pi}{n}} = w^{1/n}$


\subsection*{5.15}
Describe the conditions under which $(w^a)^b = w^{ab}$.

\textbf{Solution}

Fix a branch for $logw$. By the definition of complex power, we have for the right hand side: $w^{ab} = e^{ab \ln w}$.


For the left hand side we have $(w^a)^b = e^{b \ln w^a}$

For the two sides to much we must specify that $\ln w^a = a \ln w$

\section*{Chapter 6}

\subsection*{6.1}
Show that the Heaviside function is given by 
$$
\theta(x) = \frac{|x|+x}{2x}
$$

If $x>0$, then $\theta(x) = \frac{x+x}{2x} = 1$.

If $x<0$ then $\theta(x) = \frac{-x+x}{2x} = 0$.

\subsection*{6.2}
Show that the following function is $C^{\infty}$
\begin{equation}
    h(x) =
    \begin{cases*}
      0       & if $x \leq 0$ \\
      e^{-1/x} & if x > 0
    \end{cases*}
\end{equation}

\textbf{Sketch}
On $x>0$, $h(x)$ is a composition of smooth functions, therefore smooth (easy to show from chain rule). So if suffices to show that h is smooth at 0.

Show that all derivatives of h are sums of the form 
$e^{-1/x} \cdot (1/x)^n$

Do the substitution $u = 1/x$. Then show that each of 
$e^{-u} \cdot u^n$ goes to 0 as $u \rightarrow \infty$.

\subsection*{6.3}
Just differentiate the expansion and evaluate at 0

\subsection*{6.4}
Show that $e^{-1/x^2}$ is smooth but not analytic at 0.

\textbf{Sketch}
As in 6.2 show that all derivatives are of the form $e^{-1/x^2} \cdot 
(1/x)^n$ and show that they are all 0 at 0. Then, use an argument as in page 113 to show that if it had a Taylor expansion all coefficients have to vanish so that the function is 0 at 0.


[6.5] to [6.10] are calculus review


\end{document}