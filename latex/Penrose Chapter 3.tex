\documentclass[]{article}

%opening
\title{Road to Reality Problems and Notes\\Chapters 2 and 3}

\author{Dimitri Papaioannou}

\usepackage{amsmath}
\usepackage{mathtools}

\begin{document}

\maketitle

\section*{Chapter 2}

\subsection*{2.2}

$d(A,B) := log\frac{QA PB}{QB PA}$ 

\begin{eqnarray}
d(A,B)+d(B,C) &= log\frac{QA PB}{QB PA} + log\frac{QB PC}{QC PB}\\
&= log\frac{QA PB QB PC}{QB PA QC PB}\\
&= log\frac{QA PC}{PA QC}\\
&= d(A,C)
\end{eqnarray}


\section*{Chapter 3}
Continued Fractions is a way to expand rational numbers in a recursive form like this:

\begin{equation}
	a+\frac{1}{b+\frac{1}{c+\frac{1}{d+...}}}
\end{equation}

Unlike decimal expansions, or any base expansion, all these expressions terminate and we don't get periodic expansions for complete fractions such as
$$ \frac{1}{3} = 0.333333.... $$

Interestingly, all \textit{quadratic irrationals} have periodic expressions in this expansion, whereas other rationals have non-periodic expansions.

\subsection*{3.1}
Experiment with irrational expansions: 
https://github.com/algorythmist/continued-fractions


\subsection*{3.2}
Assuming, the following expansions continue being periodic, show that they correspond to the numbers on the left.

\textbf{Solution:}
For the first one, the periodic part satisfies the equation
\begin{eqnarray}
x &= \frac{1}{2+x} & \implies \\
x^2+2x-1 &= 0 & \implies \\
x = \sqrt{2} - 1
\end{eqnarray}

For the second, we need to show 
\begin{equation}\label{eq2}
2-\sqrt{3} = \frac{1}{3+x}
\end{equation}

where $x$ is given by the recursive relation:

\begin{eqnarray}
x &= \frac{1}{1+\frac{1}{2+x}} & \implies \\
x^2+2x-21 &= 0 & \implies \\
x = \sqrt{3} - 1
\end{eqnarray}

Plugging this back into \ref{eq2} we get
\begin{eqnarray}
2 - \sqrt{3} &= \frac{1}{1+3+(\sqrt{3}-1)} & \implies \\
2 - \sqrt{3} &= \frac{1}{2+\sqrt{3}} & \implies 
\end{eqnarray}

which is correct.

\subsection*{3.3}
The criterion that the ratio a:b is greater than the ratio c:d is that some positive integers M and N exist
such that the length of a added to itself M times exceeds b added to itself N times,
while also the length of d added to itself N times exceeds c added to itself M times. Can you see why this works?

\textbf{Solution:}
Translating the sentence above we have
$Ma > Nb$ and $Nd > Mc$

so, in other words
$$
a/b > N/M >c/d
$$

\end{document}